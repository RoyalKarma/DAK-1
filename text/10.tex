\clearpage
\section{Protichybové kódové systémy}
\subsection{Uveďte základní dvě skupiny protichybovýh kódu (PKS). Vysvětlete rozdíl mezi nimi.}
\begin{itemize}
    \item \textbf{PKS bez zpětné vazby} - Zabezpečení pomocí opravných protichybových kódů (\textbf{FEC}). Zabezpečení pomocí detekčních protichybových kódů bez zpětného kanálu (Všechny chybné přenesené zprávy jsou ztraceny). Zabezpečení pomocí protichybových kódů pracujících ve $smíšeném$, tj. opravném i detekčním režimu (Většina chybné přenesených zpráv je opravena, část chybné přenesených zpráv je ztracena.)
    \item \textbf{PKS se zpětnou vazbou} - Zabezpečení pomocí detekčních  protichybových kódů s opakováním chybného přenosu (\textbf{ARQ}) (snížení propustnosti). Smíšené zabezpečení pomocí protichybových kódů pracujících v opravném i detekčním režimu - většina chybně přenesených zpráv je opravena (\textbf{FEC}), pro neopravitelné zprávy je využit mechanizmus ARQ.
\end{itemize}

\subsection{Uveďte a stručně popište základní typy ARQ systémů}
Chyba je opravena novým bezchybným přenosem. Při tom všechny předcházející přenosy s chybou jsou považovány za nepoužitelné
\begin{itemize}
    \item \textbf{Jednobloková ARQ} - Vysílač vyšle vždy pouze jeden blok, a pak se přenos přeruší až do doby, než přijde zpětným kanálem potvrzovací zpráva o správnosti přenosu. Opakovaní přenosu/pokračování
    \item \textbf{Skupinové ARQ} - Vysílač vyšle vždy skupinu bloků, každý blok nezávisle zabezpečený detekčním kódem. Přijímač informuje zpětným kanálem vysílač o bezchybnosti celé skupiny.
    \item \textbf{Nepřerušované ARQ} - Vysílač vysílá bloky bez přerušení až do doby, kdy přijde zpětným kanálem zpráva o tom, že některý blok byl přijat s chybou. Reakce selektivní/neselektivní opakování.
\end{itemize}
\subsection{Rozdíl mezi ARQ a FEC, výhody a nevýhody.}
\textbf{FEC} + Většina chybně přenesených zpráv je opravena. - Část chybné (neopravitelné) přenesených zpráv je ztracena. 

\textbf{ARQ} + Výhodou je použití detekčních kódů, jejichž nadbytečnost je nižší než u kódu opravných. - Významnou nevýhodou je snížení propustnosti z důvodu čekání na potvrzovací zprávy. - přenosový systém schopen realizovat některou podobu zpětného kanálu.

\subsection{Co udává propustnost systému a efektivní informační rychlost?}
\begin{itemize}
    \item Rozlišujeme propustnosti:
    \begin{itemize}
        \item absolutní ($PR$) - poměr úspěšně přenesených datových jednotek a časového intervalu 
        (totožné s efektivní informační rychlostí)
        \item relativní ($PR_r$) - poměr úspěšně přenesených jednotek za časový interval k maximálnímu
        možnému přenositelnému množství datových jednotek za tento interval
        \item Datové jednotky mohou být bity, znaky, bloky, zprávy
    \end{itemize}
    \item efektivní informační rychlost - poměr úspěšně přenesených datových jednotek a časového intervalu 
\end{itemize}

\subsection{Jaký vztah je mezi kódovým poměrem a efektivní informační rychlost v ARQ systémech a
v FEC systémech.}
\begin{itemize}
    \item ARQ: $v_{ef}=PRB_r\cdot R$, kde $v_{ef}$ je efektivní informační rychlost, $PRB_r$ je relativní propustnost
    a $R$ je kódový poměr, $R=\frac{k}{n}$, kde $k$ je počet informačních bitů a $n$ je počet všech bitů
    \item FEC: $v_ef=R$, protože $PRB_r$ je díky opravám chyb rovno jedné
\end{itemize}

\subsection{Srovnejte ARQ a FEC protichybové systémy.}
\subsubsection{ARQ}
\begin{itemize}
    \item \textbf{A}utomatic \textbf{R}epeat Re\textbf{q}uest
    \item Používá detekční kódy
    \item Při detekci chyby vyšle data znovu
    \item Efektivní informační rychlost závisí na chybovosti kanálu, kódovém poměru a algoritmu ARQ 
    (Při Stop\&Wait je propustnost nízká vždy, při GoBackN je propustnost nízká u kanálu  s vysokou chybovostí.
    ARQ SelectiveRepeat je vhodné jen na zařízení s dostatečně velkou vyrovnávací pamětí)
    \item Vyžaduje zpětný kanál, kterým se vysílač dozví o nutosti opakování nebo potvrzení přijetí
\end{itemize}
\subsubsection{FEC}
\begin{itemize}
    \item \textbf{F}orward \textbf{E}rror \textbf{C}orrection
    \item Používá korekční kódy
    \item Při výskytu chyby rovnou opraví
    \item Efektivní informační rychlost závisí na kódovém poměru 
    (ale musí být vhodně zvolen, aby eliminoval chybovost kanálu)
    \item Nevyžaduje zpětný kanál
\end{itemize}

\subsection{Uveďte výhody a nevýhody systému se smíšeným zabezpečením HCS}
\begin{itemize}
    \item \textbf{H}ybrid \textbf{C}ode \textbf{S}ecurity
    \item Výhody:
    \begin{itemize}
        \item Může používat jediný kód s korekčními i detekčními vlastnostmi
        \item V případě detekce chyby je možné vyžádat si korekční doplněk (bity potřebné k úpravě kódu na korekční)
    \end{itemize}
    \item Nevýhody:
    \begin{itemize}
        \item Oproti korekčnímu režimu mívají kódy ve smíšeném režimu menší realizační zisk
        \item Pokud chceme použít kaskádově zapojené detekční a korekční kódy, musíme je vhodně volit, ať se vzájemně
        neovlivňují
    \end{itemize}
\end{itemize}