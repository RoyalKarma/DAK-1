\clearpage
\section{Stromové kódy}
\subsection{Popište základní vlastnosti stromových kódů a odlišnost oproti blokovým.}
Stromové kódy mají jiný způsob zabezpečení datového toku v porovnání s kódy blokovými. Dalším prvkem zabezpečení zde, krom vkládání nadbytečných prvků, je jejich závislost na předchozích informačních prvcích nezabezpečeného datového toku.

\subsection{Vysvětlete zadávání konvolučních kódů vytvářecími mnohočleny.}

\subsection{Vysvětlete zadávání konvolučních kódů vytvářecími maticemi.}

\subsection{Vysvětlete princip a popište stromový graf.}

\subsection{Vysvětlete princip a popište mřížový graf.}

\subsection{Vysvětlete princip a popište stavový diagram.}

\subsection{Vysvětlete princip prahového dekódování konvolučního kódu.}

\subsection{Vysvětlete princip postupného pravděpodobnostního dekódování konvolučního kódu.}

\subsection{Vysvětlete princip Viterbiho dekódovacího algoritmu.}

\subsection{Co je to délka kódového ohraničení $\nu$? Jaká je souvislost této délky a ochranné
dekódovací hloubky, při využití Viterbiho diagramu?}